\section{Testing}
In this section, I will detail the various ways I tested the project during development and the more in depth testing upon completion.
\subsection{Development Testing}
As described in the design section, the project was built in layers, and each of these could be testing individually.
\subsubsection{Proxy}
The proxy was tested in development by using it as a TCP passthrough, and sending files through it and checking the hash of the file at both sides to ensure it was the same.\par
On top of this, time-sensitive tests were run to ensure that buffering was not massively increasing latency.\par
The next tests I ran was disconnecting the sockets at both ends at different times and reopening the sockets to see if the connection could be re-established.\par
\subsubsection{HTTP layer}
Testing the HTTP layer was done by parsing a variety of HTTP requests from a browser to multiple sites and the sites responses and making sure that the layer correctly understood all of the data that it was being given.\par
Data was then manipulated in the requests to ensure that both chunked and normal encoding types were supported when the data was manipulated.
\subsubsection{Obfuscation layer}
Testing the obfuscation layer consisted of lots of tests involving large files and random data to check that data that has been obfuscated can be recovered correctly.\par
Testing the whitespace was simple, as it is a linear data transfer, and if some data gets through, it can be safely assumed that more will follow.
Testing the shuffling is slightly more difficult to test because as the shuffling algorithm is new work, there are no guarantees it would work correctly.\par It was tested with large lists and lots of data.
Testing the generic obfuscation was a multi stage process, firstly testing that simple obfuscation can work, with large random data files. Once this was fully tested, it was tested with more complex obfuscation and similar large files, finally resulting in being tested with the full HTTP obfuscation and large files.
\subsubsection{End to end testing}
The end to end testing was similar to testing the proxy, as the goal was for the behavior to be the same, namely reliable data transfer without mounting latency caused by overzealous buffering.

\subsection{Application testing}
As the project acts as a TCP tunnel, some sensible testing would be to test it with applications that make it useful.
The applications that were tested are the following:
\begin{itemize}
    \item cUrl
    \item SSH
    \item SSH Socks Proxy
    \item OpenVPN
    \item iperf
\end{itemize}

\subsubsection{Testing setup}
The client was set up on one computer, and the server was setup on a second, connected over a fast local network.
The webserver that the server connects to was also hosted on the local network.

\subsubsection{TCP Server-Server adapter}
For all of the following tests as the server exposes a TCP listening socket, and as does the service with which to connect, a bridge is needed between the two, so I made a small program which connects to both servers and forwards all data between them.

\subsubsection{Testing with cUrl}
cUrl is a program that can be used to download web pages and make HTTP requests.
The remote server was pointed at a webserver, and then curl requests could be made to the local listening port.\\
cUrl was able to get web pages of varying sizes correctly.

\subsubsection{Testing with SSH}
SSH is a program that can be used to remotely manage computers and perform tasks on them.
The remote server was pointed at an SSH server, and ssh was used to connect to the local socket and make an SSH connection.
It was possible to do server administration and even file editing with programs like \texttt{vim}, and the responsiveness that was provided was surprising, given the very limited bandwidth that was available.

\subsubsection{Testing with SSH Socks Proxy}
A feature that SSH provides is the ability to advertise a local socks proxy on the client side.
A socks proxy receives HTTP requests for remote webservers and forwards the requests on to get the response.
When testing with this, the setup was similar to the SSH test previously, except the client connected and acted as a SOCKS5 client.

The below diagram is applicable for both SSH tests, and provides a good overview of how all of the parts connect for all tests.
\begin{center}
\begin{tikzpicture}[-latex ,auto ,node distance =3 cm and 5cm ,on grid,
    semithick, state/.style={top color=white, bottom color = processblue!20, draw,processblue, text=blue, minimum width=1 cm}, line/.style={color=black,arrows=-,line cap=butt,line width=0.07cm}]
\node[state] (A)[]{HTTP-Tunnel Client};
\node[state] (B)[above=of A]{HTTP-Tunnel Server};
\node[state] (C)[left=of B]{Webserver};
\node[state] (D)[below=of A]{SSH Client};
\node[state] (E)[above right=of B]{Server-Server adapter};
\node[state] (F)[above left=of E]{SSH Server};
\path[line] (A) edge node[left] {HTTP Traffic} (B);
\path[line] (A) edge node[left] {SSH Traffic} (D);
\path[line] (B) edge node[above,rotate=90,anchor=west] {HTTP Traffic} (C);
\path[line] (B) edge node[] {SSH Traffic} (E);
\path[line] (E) edge node[] {SSH Traffic} (F);
\end{tikzpicture}
\end{center}

\subsubsection{OpenVPN}
OpenVPN is the main open source VPN server and client.\\
To test OpenVPN, it was setup and tested as a server and client pair separately to verify the configuration was correct.\par
OpenVPN was then used on top of the HTTP-Tunnel.\par
A connection was successfully established, and data is transferred, however there are a couple of issues:
\begin{itemize}
    \item OpenVPN adds substantial overhead, and the connection is incredibly slow
    \item OpenVPN sometimes detects it is being tunneled and thinks it is an attack on TCP
\end{itemize}
The first issue is to be expected, as a VPN is quite a heavy piece of software, and it was possible to access services over the VPN link.
The second issue was intermittent, and as far as I can work out is caused by a combination of a few things:
\begin{itemize}
    \item OpenVPN reads and writes packets directly to the wire, so treats TCP as a datagram based protocol rather than as a stream based protocol
    \item When the link gets saturated packets get fragmented and concatenated by my program
    \item As the packets get broken up the HMAC for each datagram becomes invalid
\end{itemize}
The errors I got when running openvpn are the following:
\begin{verbatim}
    Wed Apr  4 03:47:31 2018 127.0.0.1:48830 TLS Error: cannot locate HMAC in incoming packet from [AF_INET]127.0.0.1:48830
Wed Apr  4 03:47:31 2018 127.0.0.1:48830 Fatal TLS error (check_tls_errors_co), restarting
...
Wed Apr  4 03:49:32 2018 127.0.0.1:49474 WARNING: Bad encapsulated packet length from peer (39657), which must be > 0 and <= 1627 -- please ensure that --tun-mtu or --link-mtu is equal on both peers -- this condition could also indicate a possible active attack on the TCP link -- [Attempting restart...]
Wed Apr  4 03:49:32 2018 127.0.0.1:49474 Connection reset, restarting [0]
Wed Apr  4 03:49:32 2018 127.0.0.1:49474 SIGUSR1[soft,connection-reset] received, client-instance restarting
\end{verbatim}
They seem to be linked and caused by the issues highlighted above.\\
It may have been possible to debug and solve all of the issues, however the throughput from OpenVPN was so low, it was almost unusable so there was not much point.

\subsubsection{iperf}
Iperf is a TCP-based performance tester.
I setup a listening server on the remote end, and a client on the local server.
Over a variety of tests, I was able to get 32.7kb/s download, and 17.8kb/s upload.\\
While these speeds are not blisteringly fast, they are reasonable and roughly what I would expect as the overhead is so great.
