\section{Introduction}
\subsection{Problem Definition}
Censorship and monitoring of web traffic is a common occurrence around the world, as is deep packet inspection to ensure that traffic on port 80 is in fact HTTP traffic.
On top of this, organisations have been known to man-in-the-middle HTTPS traffic by adding their own certificates to devices owned by the organisation.
Therefore it would be useful to be able to tunnel data over HTTP and make it virtually impossible to tell that data transfer is occurring.

\subsection{Aims}
The aim of this project is to be able to tunnel data, and by extension operate a VPN over the Hypertext Transfer Protocol (HTTP). There are multiple reasons why this would want to be done, for example:
\begin{itemize}
    \item To access services that are blocked by the current network
    \item To hide the fact that blocked services are being accessed
    \item To maintain privacy regarding services that are being accessed
\end{itemize}
In order to perform the above, a server and client pair would need to be created, a server that acts as a HTTP server, and can receive connections from the client that will act as the HTTP client.
\vspace{0.5cm}
In formal terms, the aims of the project are as follows:
\begin{enumerate}
    \item To hide data in the HTTP requests and responses
    \item For the server to be able to function as a valid HTTP Server
    \item To be able to communicate between the server and client using only HTTP
    \item To make the traffic appear as valid HTTP
\end{enumerate}
For example, it could be valuable to be able SSH to a remote server, and make it very difficult for an onlooker to tell that:
\begin{itemize}
    \item Data Transfer is occurring
    \item That the traffic is not HTTP
    \item That other parties are involved in the connection
\end{itemize}
The benefit of the server acting as a valid HTTP server is because if it is being investigated, the server could be browsed to, and it would not be obvious that it is not a real HTTP server, or that at least HTTP is not the only function of the server.

\subsection{Paper Overview}
\begin{enumerate}
    \item Introduction
        \begin{itemize}
            \item Summary of the aims the project intends to fulfill
            \item Short overview of the project
        \end{itemize}
    \item Existing Work
        \begin{itemize}
            \item Review of literature surrounding the project
        \end{itemize}
    \item Background
        \begin{itemize}
            \item High level overview of concepts and designs essential for the project:
                \begin{itemize}
                    \item HTTP
                    \item TCP
                    \item VPN
                    \item Privacy
                \end{itemize}
        \end{itemize}
    \item Specification
        \begin{itemize}
            \item Project specification
        \end{itemize}
    \item Design
        \begin{itemize}
            \item High level program architecture and design
            \item Configurable options
            \item Design choices
        \end{itemize}
    \item Implementation
        \begin{itemize}
            \item Detailed information about how each component functions
            \item High level overview about data flows
        \end{itemize}
    \item Testing
        \begin{itemize}
            \item Testing methodologies
            \item Testing strategy
        \end{itemize}
    \item Project Management
        \begin{itemize}
            \item Explanation of how the project was managed
        \end{itemize}
    \item Discussion
        \begin{itemize}
            \item An overview of about what was successful
            \item A review of what was learned
        \end{itemize}
    \item Conclusion
        \begin{itemize}
            \item Compare the project to the aims
            \item Conclude whether or not the project was successful
        \end{itemize}
\end{enumerate}


